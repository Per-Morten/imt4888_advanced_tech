%% This document gives an example on how to use the ntnumasterthesis
%% LaTeX document class.

%% Use short name MACS, MIS, CIMET, MTDMT, MIXD or MIS
%% Language english or norsk
%% b5paper with oneside or twoside, you can set A4 if you want but you submit in b5

%% If you want print with the heading material on a4 paper you can use this format
%% \documentclass[MACS,english,a4paper,oneside,12pt]{ntnuthesis/ntnuthesis}

%% with the change to using DAIM we have a new option. include DAIM after english below removes the front page material so that you can then submit in the DAIM system. If you are wanting the front material remove DAIM and make sure you fill in the DaimData.tex file.
\documentclass[MACS,english]{ntnuthesis/ntnuthesis}

\usepackage[T1]{fontenc}
\usepackage[utf8]{inputenc}     % For utf8 encoded .tex files allows norwegian characters in the files. This can be dangerous if you change to a differnt editor.
%\usepackage[pdftex]{graphicx, hyperref}   % For cross references in pdf
\usepackage{graphicx}
\usepackage{hyperref}   % For cross references in pdf

% For smart references
%    use \cref{label} and Caption and Number will be added automatically
\usepackage[capitalise,noabbrev]{cleveref}

\usepackage[dvipsnames]{xcolor}              % For colouring text
\hypersetup{colorlinks=true,
		linkcolor=blue,          % color of internal links (change box color with linkbordercolor)
    citecolor=blue,        % color of links to bibliography
    filecolor=blue,      % color of file links
    urlcolor=blue           % color of external links
		}
\usepackage{csvsimple}  % for simple table reading and display
\usepackage{url}
\usepackage{booktabs}
\usepackage{gnuplottex} %miktex option if using miktex on windows
\usepackage{rotating}
\usepackage{float}


\definecolor{darkgreen}{rgb}{0,0.5,0}
\definecolor{darkred}{rgb}{0.5,0.0,0}

\lstset{        basicstyle=\ttfamily,
                keywordstyle=\color{blue}\ttfamily,
                stringstyle=\color{darkred}\ttfamily,
                commentstyle=\color{darkgreen}\ttfamily,
}



\usepackage{listings}
\usepackage{xcolor}

\definecolor{dkgreen}{rgb}{0,0.6,0}
\definecolor{lightgray}{rgb}{0.975,0.975,0.975}
\lstdefinelanguage{csharp}{
      backgroundcolor=\color{lightgray},  
      basicstyle=\footnotesize \ttfamily \color{black} \bfseries,   
      breakatwhitespace=false,       
      breaklines=true,               
      captionpos=b,                   
      commentstyle=\color{dkgreen},   
      deletekeywords={...},          
      escapeinside={\%*}{*)},                  
      frame=lines,                  
      language=C,                
      keywordstyle=\color{purple},  
      morekeywords={BRIEFDescriptorConfig,string,TiXmlNode,DetectorDescriptorConfigContainer,var,private,public,Vector3,Mesh}, 
      identifierstyle=\color{black},
      stringstyle=\color{blue},      
      numbers=left,                 
      numbersep=5pt,                  
      numberstyle=\tiny\color{black}, 
      rulecolor=\color{black},        
      showspaces=false,               
      showstringspaces=false,        
      showtabs=false,                
      stepnumber=1,                   
      tabsize=4,                     
      title=\lstname,                 
}
%\usepackage{listings}
\usepackage{xcolor}

\definecolor{dkgreen}{rgb}{0,0.6,0}
\definecolor{lightgray}{rgb}{0.975,0.975,0.975}
\lstdefinelanguage{csharp}{
      backgroundcolor=\color{lightgray},  
      basicstyle=\footnotesize \ttfamily \color{black} \bfseries,   
      breakatwhitespace=false,       
      breaklines=true,               
      captionpos=b,                   
      commentstyle=\color{dkgreen},   
      deletekeywords={...},          
      escapeinside={\%*}{*)},                  
      frame=lines,                  
      language=C,                
      keywordstyle=\color{purple},  
      morekeywords={BRIEFDescriptorConfig,string,TiXmlNode,DetectorDescriptorConfigContainer,var,private,public,Vector3,Mesh}, 
      identifierstyle=\color{black},
      stringstyle=\color{blue},      
      numbers=left,                 
      numbersep=5pt,                  
      numberstyle=\tiny\color{black}, 
      rulecolor=\color{black},        
      showspaces=false,               
      showstringspaces=false,        
      showtabs=false,                
      stepnumber=1,                   
      tabsize=4,                     
      title=\lstname,                 
}

%Typesetting of C++ but not always stable in titles etc...
\newcommand{\CPP}[0]{{C\nolinebreak[4]\hspace{-.1em}\raisebox{.1ex}{\small\bf +\hspace{-.1em}+\ }}}

%\usepackage[table]{xcolor}% http://ctan.org/pkg/xcolor
%\usepackage[nomessages]{fp}
%\newlength{\maxbarlen}


\newcommand\databar[3][gray!20]{%
  \FPeval\result{round(#3/#2:4)}%
  \rlap{\textcolor{#1}{\hspace*{\dimexpr-\tabcolsep+.5\arrayrulewidth}%
        \rule[-.05\ht\strutbox]{\result\maxbarlen}{.95\ht\strutbox}}}%
  \makebox[\dimexpr\maxbarlen-2\tabcolsep+\arrayrulewidth][r]{#3}}

\newcommand{\todo}[1]{{\color{blue}TODO: #1\\}}
%\renewcommand{\todo}[1]{}

\newcommand{\rephrase}[1]{{\color{Aquamarine} #1}}
%\renewcommand{\rephrase}[1]{#1}
\newcommand{\think}[1]{{\color{Orchid} #1 }}
%\renewcommand{\think}[1]{}


\newcommand{\com}[1]{{\color{red}#1}} % supervisor comment
%\renewcommand{\com}[1]{} %remove starting % to remove supervisor comments
% This will appear in text \com{Lecuters comment} and be visible unless you uncomment
% the renewcommand line.

%\newcommand{\todo}[1]{{\color{green}#1}} % items to do
%\renewcommand{\todo}[1]{} %remove starting % to remove items to do

\newcommand{\n}[1]{{\color{blue}#1}} % other comment
%\renewcommand{\n}[1]{} %remove starting % to remove notes

\newcommand{\dn}[1]{} % add the d to a note to say that you have finished with it.




% Set to true ONLY if using Harvard citation style
\newboolean{HarvardCitations}
\setboolean{HarvardCitations}{false} % false for computer science, true for interaction design and harvard style


\ifthenelse{\boolean{HarvardCitations}}{%
	\usepackage{natbib} % for Harvard names as citations.
}{%
	\usepackage[numbers]{natbib} % for Vancover numbers in bibliography
}

\newcommand{\q}[1]{\leavevmode\marginpar{\small\em #1}}
\renewcommand{\q}[1]{}


\begin{document}

\input{inc/DaimData} % this is the file which contains all the details about your thesis
\makefrontpages % make the frontpages
\input{inc/mastersIntro}

\tableofcontents

\hypersetup{pageanchor=true}

% Comment with a percent to remove figures or tables:
%\listoffigures
%\listoftables
%\lstlistoflistings

\think{
    Possible topics to cover:
    \begin{itemize}
        \item Human muscles can be modelled as spring system, brain sets desired spring length.
        \item Hybrid systems (Raw spring mass in just cause, but it is further fixed with algorithms)
        \item Give an overview of the prep (Do in reflection)
        \item Cover mesh deformation as springs
        \item Cover mesh manipulation as physical objects with constraints (cloth)
        \item Bring up papers that I was pointed at which was hard to understand.
        \item Comment on both kinetic and dynamic (don't need cover)
        \item Cover Doing it on the CPU, doing it on the GPU (in Shaders) (don't need to cover)
    \end{itemize}
}


\todo{Ensure that I always say: mass spring systems}

\think{Add in that shape generation can be a smart thing to do to get a better overview of how everything fits together.}
\think{Should I focus more on the usage of mass spring models, and discuss what they can be used for, i.e. not just going for mesh manipulation, but also other types of physical simulations?}

\todo{Add link to repo! Invite Simon to the Repo}


\think{
Structure:
Introduction
Problem Description
Simple Mesh Manipulation like Rounding a cube (Done both on CPU and GPU) % Don't do this?
Spring Mass Systems
What can Spring Mass Be Used For?
Sources
* Include not just sources and good tutorials, but perhaps also that it can be wise to read up on mesh generation.
Reflection
}

\chapter{Introduction}
\label{chap:introduction}
Mesh manipulation is a broad area, and is done for many different reasons.
Mesh manipulation can be used for collision deformation, animation, physics simulation and the likes. 
However, mesh manipulation can be quite an intimidating to approach, due to the mathematical intensity of the field.
This paper presents a discussion on mesh deformation through spring-mass systems, and also presents two implementations for mesh deformation,
using the spring mass damper model. One with semi-implicit euler integration, the other using verlet integration and constraints.

\think
{
    \begin{itemize}
        \item Done for various reasons: Collision response, animation, physics emulation.
        \item Exist many solutions, with different levels of accuracy
        \item Field can be intimidating due to all the math involved
        \item Many solutions are really accurate, working well in film or non-realtime simulation
        \item Games often need to sacrifice accuracy for speed.
    \end{itemize}
}
 % includes latex files from the same directory
\chapter{Problem Description}


\think
{
    \begin{itemize}
        \item Done for various reasons: Collision response, animation, physics emulation.
        \item Exist many solutions, with different levels of accuracy
        \item Field can be intimidating due to all the math involved
        \item Many solutions are really accurate, working well in film or non-realtime simulation
        \item Games often need to sacrifice accuracy for speed.
    \end{itemize}
}
 % could be called Methodology or methods or any filename
\chapter{Implementation}
\section{Background Theory}
\label{chap:implementation}
When implementing a real-time mesh deformation system certain decisions must be made.
Firstly one must decide upon the model to use for the mesh deformation, the more brute-force mass-spring system or the algorithmic approach using proper physics.
As previously indicated we are working with a mass-spring system for this implementation.

Following the decision of models we need to decide on an integration scheme, how should we model the changes in acceleration, position, and velocity of our vertices?
Many different integration schemes exits, such as explicit Euler integration, Runge Kutta, or Verlet integration, all with different properties and difficulty of implementation.
The two used in the following implementations will be semi-implicit Euler integration and Verlet integration.

Lastly, we need to decide if we go for a kinematic system where we model springs as constraints, 
or a dynamic system where we model the springs as proper springs working with Hooke's law\cite{math_for_games}, both of which are presented within this chapter.

\subsection{What is Semi-Implicit Euler Integration?}
Semi-Implicit Euler integration is perhaps the integration scheme that is easiest to follow and comes most natural to programmers when they implement
movement in a real-time system for the first time. It is also a scheme used within a lot of physics engines~\cite{gafferongames_integration}.

In Semi-Implicit Euler integration we keep the total accumulated force, velocity and position of each game object.
Upon each time-stepped update of the system we divide all the force that has been applied to an object by the mass of the object to get the acceleration of the object.
Multiplying this acceleration with the timestep yields the change in velocity of the object, which added on the previous velocity becomes the new velocity.
This new velocity is then multiplied with the timestep and added to the position, reflecting the change in position over time.
The code for this can be seen in Listing~\ref{code:semi-implicit_euler_integration}.

\begin{figure}
\begin{lstlisting}[label={code:semi-implicit_euler_integration},language=csharp,caption={Semi-Implicit Euler Integration}]
private void SemiImplicitEuler(float dt, GameObject go)
{
    var acceleration = go.force / go.mass;
    go.velocity += acceleration * dt;
    go.position += go.velocity * dt;
}
\end{lstlisting}
\end{figure}

\subsubsection{Advantages \& Disadvantages}
The largest advantage of the semi-implicit Euler integration is its ease of implementation, as well as the generally low computational complexity.
However, it is not entirely stable, meaning that you can get into situations where your numbers start exploding.

\subsection{What is Verlet Integration?}
Rather than storing velocities and positions of each game object like the Euler integration, Verlet integration stores the current and previous positions of each game object.
Per timestep, the current velocity is calculated implicitly by subtracting the current position from the previous one.
Changes to the current velocity are also modeled via positional data by integrating the game object's acceleration twice over the timestep.

The new position of a game object following a Verlet integration is then: The current position added with the calculated velocity and the positional change due to acceleration.
Code for this can be seen in Listing~\ref{code:verlet_integration}.

\begin{figure}
\begin{lstlisting}[label={code:verlet_integration},language=csharp,caption={Verlet Integration}]
private void Verlet(float dt, GameObject go)
{
    var acceleration = go.force / go.mass;
    var curr = go.pos + (go.pos - go.prevPos) + acceleration * dt * dt;
    go.prevPos = go.pos;
    go.pos = curr;
}
\end{lstlisting}
\end{figure}

\subsubsection{Advantages \& Disadvantages}
Just like semi-implicit Euler integration this integration model is relatively easy to implement, and relatively fast.
Due to the velocity being calculated based on the positions, it is very hard to end up in a situation where the velocities and positions come out of sync,
meaning that the model is quite stable.
I would, however, argue that while it is easy to implement it is not as easy to understand as the Semi-Implicit Euler integration, at least upon the first contact with the model.
I needed a proper intuitive explanation from Mosegaard\cite{mosegaards_clothing_simulation} before I understood what was really going on.
Additionally, Verlet integration, at least with this implementation requires a fixed time step, otherwise, the velocity calculation won't be correct~\cite{alan_wake_mass_spring}.

\subsection{Kinematic or Dynamic approach?}
When working with mesh deformation one needs to decide upon a kinematic or a dynamic approach for controlling the springs.

Within a kinematic approach, the springs between the vertices are thought of as constraints that you need to solve for.
Essentially this means that if two points are too far apart from each other, you move them a bit closer together, potentially over several iterations to ensure that you are not creating
new inconsistencies in the model.
The kinematic approach is the easiest to implement of the two and can never explode, however, the extra iterations needed to solve the constraints can have performance implications.

The dynamic approach on the other hand models the springs as actual springs that correct themselves through forces according to the laws of physics, such as Hooke's Law.
This approach is much harder to implement and can lead to situations where the model spirals out of control (i.e. explodes).
Additionally, while implementing soft springs in with this approach is not that difficult, harder springs require more complicated integration schemes\cite{math_for_games}.

While both of these approaches imply different terminology for the connections/springs within a system I will refer to the connections between vertices as springs for the rest of the article, as the name of the model is "mass-spring system".

\section{Implementation}
Once all the decisions of the deformation model, integration scheme and kinematic/dynamic approach has been made it is possible to start the implementation of mesh deformation.

\subsection{Dynamic Approach - Ball Deformation}
Jasper Flick\cite{catlike_mesh_deformation} shows an implementation\footnote{Video of this implementation in action: \url{https://www.youtube.com/watch?v=f-5RzfN8kjw}} where the springs are implied based on the original position of each vertex and their new position.
The implementation takes a dynamic approach to deformation and follows a semi-implicit Euler integration scheme.

In this implementation, we keep a copy of the initial vertex positions, as well as a collection of the modified vertices.
Additionally, we keep a collection of the velocity of each vertex, which we update each time force is added to a vertex.

The springs come into play when updating the positions of all the vertices based on their velocities.
Each spring within this system is the vector between the initial positions of the vertices in the mesh and their current positions, as seen in Figure~\ref{fig:catlike_mesh_deformation_springs}.
Upon update of the vertices, the force of a spring is applied to the velocity of a vertex in the direction of the spring, moving the vertex back towards its initial position.
The code for this can be seen in Listing~\ref{code:catlike_mesh_deformation_update}.

\begin{figure}
\centering
    \scalebox{0.5}{
    \includegraphics[width=\textwidth]{report/figures/catlike_mesh_deformation_springs.png}
    }
    \caption{Modified vertices(yellow) are pulled towards their original position(blue) by the springs (pink)\cite{catlike_mesh_deformation}}
    \label{fig:catlike_mesh_deformation_springs}
\end{figure}

\begin{figure}
\begin{lstlisting}[label={code:catlike_mesh_deformation_update},language=csharp,caption={Catlike coding mesh deformation vertex update}]
private void UpdateVertex(int i)
{
    Vector3 velocity = mVertexVelocities[i];
    Vector3 spring = mDisplacedVertices[i] - mOriginalVertices[i];
    spring *= UniformScale;
    velocity -= spring * SpringForce * Time.deltaTime;
    velocity *= 1f - Damping * Time.deltaTime;
    mVertexVelocities[i] = velocity;
    mDisplacedVertices[i] += velocity * (Time.deltaTime / UniformScale);
}
\end{lstlisting}
\end{figure}

\subsubsection{Usage}
As seen in the video, although simple, the implementation provides quite a lot of flexibility, toying with the different variables
can lead to some interesting visual effects.
For example, applying negative force to the sphere can be used to create a visual effect similar to that of a star being swallowed by a black hole\footnote{See: \url{https://youtu.be/f-5RzfN8kjw?t=62}}.
Applying outward force from the inside of the model can create a relatively convincing effect of something moving around inside the object\footnote{See: \url{https://youtu.be/f-5RzfN8kjw?t=28}}.
Giving the sphere a high SpringForce will lead to it being harder to deform, and more bouncy when trying to return to its original shape.

\subsubsection{Limitations}
As mentioned by Jasper Flick~\cite{catlike_mesh_deformation} this implementation is not a physics simulation, while the mesh is deformed,
the colliders and physical representation of the object stay the same.

Additionally, none of the vertices are connected to each other through springs, and therefore move completely independent from each other.
This means that if we were to for instance pin two of the vertices to their initial position, and then apply force to the mesh
none of the other vertices would respect that two of them were locked in place, leading to less than satisfying results.

\subsection{Kinematic Approach - Cloth Simulation}
Cloth simulation is a place where the springs are more apparent than the previous implementation\footnote{Video of cloth simulation: \url{https://www.youtube.com/watch?v=SY9PPEl8Kmo}}. 
The implementation is based on that of Jesper Mosegaard\cite{mosegaards_clothing_simulation} and Thomas Jakobsen\cite{jakobsen_advanced_character_physics}.
Unlike the previous implementation, the springs here are explicit and ensure that vertices are not handled in isolation but rather are affected by the other vertices in the system.
This gives greater control in that it allows you to apply force to a specific vertex and the whole system responds, rather than having to apply the force to all the vertices.

\subsubsection{Spring Types}
There are three types of springs within this implementation, all serving the same purpose of holding the model together,
but in different ways. The different types can be seen in Figure~\ref{fig:spring_types}.
\begin{figure}
    \centering
    \caption{The different spring types (here labeled constraints). Image taken from Jesper Mosegaard\cite{mosegaards_clothing_simulation}}
    \scalebox{0.5}{
    \includegraphics[width=\textwidth]{report/figures/spring_types.png}
    }
    \label{fig:spring_types}
\end{figure}

\paragraph{Structural Springs}
The structural springs are the vertical and horizontal connections between the vertices, which ensures that the model stays in one piece.
However, they alone are not enough, as the model can collapse into itself in a two dimensional space\cite{jeff_lander_real_time_cloth} this can be seen in Figure~\ref{fig:structural_springs_collapsing}.

\begin{figure}
    \centering
    \caption{Cloth of only structural springs in the process of collapsing into itself}
    \scalebox{0.5}{
    \includegraphics[width=\textwidth]{report/figures/structural_collapse.png}
    }
    \label{fig:structural_springs_collapsing}
\end{figure}

\paragraph{Shear Springs}
Shear springs fix the issue mentioned above by connecting each vertex to their diagonal neighbors.
When the vertices are about to start collapsing in on themselves the shear springs will push them apart again\cite{jeff_lander_real_time_cloth}.
With this, the model also acts in a more three-dimensional space, as the vertices need to make use of the third dimension to satisfy the shear springs, as seen in Figure~\ref{fig:shear_springs_collapsing}.
\begin{figure}
    \centering
    \caption{Cloth of structural and shear springs, preventing it from collapsing into 2D space}
    \scalebox{0.5}{
    \includegraphics[width=\textwidth]{report/figures/shear_collapse.png}
    }
    \label{fig:shear_springs_collapsing}
\end{figure}

\paragraph{Bending Springs}
Technically you only need these structural and shear springs to create an ok looking cloth simulation,
however, a third type of spring called a bending spring can be added.
These are connected between each vertex and their second neighbor,
which can help fix cases where the second neighbor of a vertex is unrealistically placed in the system.
This, in general, leads to the cloth becoming more flat, as extreme differences between the vertices are corrected for
through the second neighbouring vertices\cite{jeff_lander_real_time_cloth}.
While not as obvious, this can be seen in Figure~\ref{fig:bending_springs_collapsing}.
\begin{figure}[H]
    \centering
    \caption{Cloth of structural, shear, and bending strings, preventing it from bending unrealistically into itself.}
    \scalebox{0.5}{
    \includegraphics[width=\textwidth]{report/figures/bending_collapse.png}
    }
    \label{fig:bending_springs_collapsing}
\end{figure}

% This might be what is creating wrinkles?
% Double check that this is actually the case.
% Without
%
% 1 -- 2
%      |
%      3
% With
% ------------|
% 1 -- 2 ---| |
%           3 -

\subsubsection{Implementation}
In this implementation we keep all the springs within a collection, each spring knowing which two vertices they are connected to and the desired rest length of the spring.
Upon update, we iterate through all the vertices and calculate their new positions based on the Verlet integration scheme.

Following the update is fixing the constraints. This is done by iterating through all the springs and correcting the distance between the vertices it is connected to.
This needs to be done several times as to avoid the model seeming to elastic.
The code for this can be seen in Listing~\ref{code:satisfy_constraints}.

\begin{figure}
\begin{lstlisting}[label={code:satisfy_constraints},language=csharp,caption={Semi-Implicit Euler Integration}]
private void FixedUpdate()
{
    for (int step = 0; step < ConstraintIterations; step++)
    {
        for (int i = 0; i < mSprings.Count; i++)
        {
            var spring = mSprings[i];
            var diff = mPositions[spring.V2Idx] - mPositions[spring.V1Idx];
            var dist = diff.magnitude;
            var correction = (diff * (1 - spring.RestDistance / dist)) * 0.5f;

            mPositions[spring.V1Idx] += correction;
            mPositions[spring.V2Idx] -= correction;
        }
    }
}
\end{lstlisting}
\end{figure}

\subsubsection{Usage}
Due to its more sophisticated nature, this implementation offers more flexibility than the previous one.
Simulations of entire pieces of clothes like curtains or flags are obvious ways of using this model.
However, it is not a requirement of the model that every vertex in a mesh are simulated and part of the mass-spring system, you can also just simulate parts of it.
An example here is if you have a character wearing a dress, you probably do not need that every part of the dress
has realistic cloth physics, it is probably enough that just the bottom of the dress reacts to the environment
to create a believable appearance.

Another use case with this model could be to have a cloth that tears at a certain point because it is stretched too much,
or that parts of the cloth are cut off. This should not be too difficult, you would need to remove the springs between
the vertices that have been disconnected from the rest of the model.
Additionally, you would have to ensure that all the cuts were in line with the triangles of the mesh.

\subsubsection{Issues}
Although there are many advantages of this model, like how it is relatively easy to implement, gives quite good results and cannot "explode", it does suffer from a few issues.
A problem with this implementation is that it can get into unsolvable states.
I.e. when you apply forces to model continuously it can get into a situation where it never rests.
Such a situation can be seen in the video of the implementation after around 25 seconds\footnote{See: \url{https://youtu.be/SY9PPEl8Kmo?t=25}}.
Here the forces of gravity are continuously being applied to the vertices, in the real world the bottom
of this cloth would stop moving after a while. However, in this implementation, the cloth never stops moving
because the springs never manage to get into a situation where they all are "satisfied".
This is a weakness of the kinematic approach which was chosen for this implementation\cite{math_for_games}.

The number of times you need to iterate over the constraints to satisfy them is also an issue.
Too few iterations and the cloth will look too elastic, too many iterations and it can seem too hard,
additionally, the more iterations you add, the more processing time is needed, which can become a performance problem.

\subsubsection{Extensions}
Matthew Fisher\cite{matthew_fisher} suggests some extensions to the model,
such as quilting. Currently, the cloth looks like infinitely thin silk, which is not particularly realistic,
quilting would add thickness to the model, allowing it to look and act more like wool or cotton. 
Fisher suggests doing this with a marching cubes algorithm.
However, that would potentially be more costly to simulate and to draw.

Another extension could be self-collision or collision with other soft bodies, however, this is very challenging to implement
and quickly ends up being really expensive in terms of performance as well.

Moving these calculations to the GPU is also a possibility, as a lot of the calculations are data parallel and would probably benefit quite a lot for being executed in a parallel environment.

This also implementation does not make any distinction between the different types of springs, extending the implementation to react differently based on the different spring types could lead to more realistic behavior.

\chapter{Users}
\rephrase{While cloth simulation and simple deformation effects are obvious applications of the spring mass system, 
it has many more interesting applications.}
One example is to model rope using a spring mass system. The rope can be modelled as a collection of particles
connected together through springs which can stretch and bend to a specified degree.
The same line of thinking could then be extended to hair simulation.

Spring mass systems are also used outside of the video game world.
For example, Moosegaard~\cite{mosegaards_clothing_simulation} shows how a GPU accelerated spring mass system is suitable 
for surgical simulations. Moving the simulation over to the GPU allowed them to reach much faster convergence rates for
larger and more detailed organ models, leading to more realistic simulations of organs being cut and morphed.
Similarly, Nedel~\cite{nedel_muscle_spring_mass} shows simulations of muscle deformations using a spring mass system,
introducing angular constraints to avoid situations where a model can completely change shape.

Müller\cite{muller_fem} also brings up how simulators like these could be valuable for animators on projects where
rendering takes a long time, as it would allow them to iterate on their ideas before sending them off to hour long rendering processes.

Moving back to the video game world Jakobsen~\cite{jakobsen_advanced_character_physics} proposes using a spring mass system
for modelling "skeletons" of in-game characters. 
He describes how the corpses within the game Hitman: Codename 47 consists of just particles and springs, as seen in figure~\ref{fig:spring_mass_skeleton}.
These springs have varying degrees of stiffness to ensure that the body does not bend or rotate in unnatural ways.
For example, a very stiff spring between the legs can ensure that legs never cross over.
Additionally, Jakobsen uses angular springs to ensure that the body does not move in ways that the human body does not allow.

This approach allows for intuitive physical responses as well, an example given by Jakobsen is that if a character is hit in the shoulder
reacting to that hit can be done simply by moving the shoulder particle based on the force from the hit.
Moving the particle will impact the other springs attached to it, which will drag the shoulder and its connected particles closer to each other,
leading to realistic looking physical behavior.

\begin{figure}
    \centering
    %\includegraphics{}
    \caption{Particle/Stick figure used to represent human anatomy (Re-drawn based on Jakobsen's drawing\cite{jakobsen_advanced_character_physics})}
    \label{fig:spring_mass_skeleton}
\end{figure}

\todo{Essentially you can use spring mass systems every time you need to model some behavior where:
    Application is interactive, so you cannot accept frame drops,
    %the 
}

\todo{Add here the note that Muller brings up, that it can be good for artists in an animated movie to be able to see their resultsin real time, before it is sent off to hour long renderings}


\todo{In Real Time Muscle Deformations Using Mass-Spring Systems, they mention biomechanical model for muscle simulation, but it is too expensive.}



\chapter{Future}

\todo{Hybrid Systems?, (D4MD – Deformation system for a vehicle simulation game) }
* Eulerian-on-langrangian cloth simulation: \url{https://drive.google.com/file/d/1l3zUKHi1TVbmF_SweB6FB3xCa5XBQ6mF/view}
* Nonsmooth Developable Geometry for Interactively Animating Paper Crumpling: \url{https://dl.acm.org/citation.cfm?doid=2870647.2829948}


\chapter{Sources}
As mentioned in the introduction getting into this field can be quite intimidating,
while there are numerous papers on the topics they are often quite heavy on the math side,
at the same time I find that there are a lack of good tutorials, and the good ones are really hidden away.
The assumption seems to be that if you are interested in implementing this, 
you already know enough math and physics to understand the papers.
A lot of this stuff is explained through mathematical models which aren't necessarily the most intuitive,
such as those found in Provot's paper\cite{provot_mass_spring} on calculating the internal force of tensions between springs.
Now that I have implemented a mass-spring cloth model I understand more of what they are trying to say here,
but initially most of that stuff was just scary.
Similarly a lot of important aspects simply were not mentioned in a lot of the different papers and tutorials,
like how it is preferable that the timestep is in the integration methods are constant.
This is probably obvious to a lot of the people working in this field, but to newcomers it might not be,
and it is something I randomly figured out myself (only to later find it in a tutorial).

It can also be very confusing to get into the field if you do not know exactly what you are looking for. 
I did not know that modelling of objects for animations etc, also could be called deformation.
Now it makes a lot of sense, but when I was initially looking for deformation stuff I was looking at it from a more visual effect
viewpoint. This made it really hard to discern what papers on deformation were relevant and not.

\section{Resources}
Luckily some really good tutorials and resources exists for this stuff, however they can be a bit hard to find.
Below are some that helped me with the implementations, or that I found after completion that probably would have helped me.

\paragraph{Jasper Flick}
Jasper Flick's tutorials on Catlike Coding\footnote{\url{https://catlikecoding.com/unity/tutorials/}}
are a really good starting point. His mesh deformation tutorial gives a good introduction to mass spring systems, and deals with the problems of scale and dampening.
Additionally, I would advice anyone who is getting into mesh manipulation to go through his procedural meshes tutorial,
It allows you to brush up on how meshes actually work, and get me into the mindset of mesh manipulation.
Jasper Flick also has several other gread tutorials on manipulating meshes for visual effects, such as creating good looking waves\footnote{\url{https://catlikecoding.com/unity/tutorials/flow/waves/}}, which is what I would look at next if time would have allowed it.

\paragraph{Jeff Lander} 
Jeff Lander's article was also a good "tutorial", however I struggled when trying to adapt the code from his article\cite{jeff_lander_real_time_cloth} into Unity,
I ended up with a model that exploded really quickly, which I believe might be due to the semi-implicit Euler integration and that I was not using fixed time steps,
as I did not know of it at the time. I would like to go back to that implementation in the future to give it a spin. 
Unfortunately the links to his code on gamasutra are dead, however, the code is available online\footnote{\url{http://www.darwin3d.com/gamedev.htm}}.

\paragraph{Jesper Mosegaard}
Jesper Mosegaards tutorial\cite{mosegaards_clothing_simulation} was the one that helped me the most in understanding how cloth simulation through spring mass systems worked.
He highlights the importance of the Verlet integration and also explains it in a quite intuitive manner, additionally his code
is relatively easy to follow, however, I wish he mentioned in his article that the timestep had to be a constant, because that was something I figured out
after having followed his implementation but got inconsistent results.

\paragraph{Thomas Jakobsen}
Reading Thomas Jakobsen's\cite{jakobsen_advanced_character_physics} article is also recommended, as it goes deeper into Verlet integration and handling constraints,
and other places where spring mass systems can be used. However, this one gets a bit more advanced after a while.

\paragraph{Henrik Enqvist}
The article by Henrik Enqvist\cite{alan_wake_mass_spring} on cloth in Alan Wake is something I came upon just recently, but I wish I had found it sooner.
It gives a good overview of spring mass systems and Verlet integration, and the one who confirmed to me that fixed timestep was actually needed.
It also comes up with solutions to some of the problems I have in my own model, such as too much stretching and rubbery feel.

\paragraph{Math as Code}
Math as code\footnote{\url{https://github.com/Jam3/math-as-code}} is a great resource to have when approaching math heavy topics like this from a programmers perspective.
Essentially it covers a lot of common mathematical symbols and expresses them in Javascript. 
Having this resource gave me a lot of confidence as it allowed me to take complicated looking math and move it to a domain I was more familiar with.
For instance it was when I used math as code to "translate" the net force calculation shown by Matthew Fisher\cite{matthew_fisher} that I got the courage to give implementing cloth physics a shot.



\chapter{Reflection}
Working on this project has been an enlightening experience, but it has not been without issues.
Originally I wanted to look at shaders in Unity to brush up on my shader knowledge, with the final goal being to end up in the area of terrain deformation.
I spent the beginning of the project toying with Unity's shader graph\footnote{\url{https://blogs.unity3d.com/2018/02/27/introduction-to-shader-graph-build-your-shaders-with-a-visual-editor/}}, creating different visual effects\footnote{Video of some of two of the visual effects in shader graph: \url{https://youtu.be/1DBmfpjmMds}} to verify that my assumptions of how shaders actually worked were correct. 

After getting my feet wet in shader graph I moved over to writing the shaders by hand while looking for tutorials on terrain deformation.
It was hard to find good tutorials on this, especially on the GPU. However, I did stumble across a seemingly good tutorial from Jasper Flick\footnote{\url{https://catlikecoding.com/unity/tutorials/advanced-rendering/surface-displacement/}} 
on surface displacement which seemed quite close to what I wanted to do.
This tutorial assumed that you were already quite familiar with rendering concepts and built on his previous tutorials, so I started going through those.
While working through these tutorials I felt that Unity's shaders were too different from those I had written before in OpenGL, the learning curve
was a bit too steep, debugging tools were quite lacking, and that I was not making good progress towards the end goal.

I looked around for alternative solutions. Writing shaders in Unreal was a possibility, but that requires building the entire engine from source
when you add your own shading models\cite{unreal_shaders}, and the compilation times would quickly hurt my iteration process.
Another alternative was to implement my own rendering program in C++, but that would have taken too much time.

Due to this I decided to move back to the CPU, and work on more general mesh deformation there.
The original intention was to port the logic over to vertex shaders after I had implemented them on the CPU.
I soon moved away from that idea. Implementing stuff on the GPU probably would be more of a chore than a learning experience,
as I already have dealt with data layout and CPU-GPU communication in other courses.
The challenge would be the logic of mesh deformation, not how to do it in shaders.

With the shift over to the CPU, I started to work on procedural mesh generation, as all the different tutorials I looked at
on mesh deformation was working on procedurally generated meshes. Working through the procedural mesh generation tutorials
of Jasper Flick was quite intuitive and gave results, and in the end, I finished his mesh deformation tutorial.
I played around with the implementation for a while, without a clear goal of where to next.

Looking through papers on the topic of mesh deformation most of it seemed to focus on deformation for manipulating meshes more related to creating animations,
which was not really what I was after. 
It was first when I learned that the mesh deformation that Jasper Flick\cite{catlike_mesh_deformation} presented was a sort of mass-spring system
that I figured out that deformation through mass-spring systems would be what I wanted to focus on, and its use in cloth simulation was quite enticing.
Unfortunately, this was quite late in the course, so I did not get to spend as much time on the background of the technology, alternatives
and the future of the technology as I originally wanted.
I would have loved to get more into the other approaches like the finite element methods or free form deformations or try the mass-spring system
within other contexts or with other integrators, moving more into the area of physics simulations.
Figuring out the topic so late also meant that I ran into the problem that you do not really know what to search for until you have been in an area for a while,
which meant that looking around for more background papers and tutorials was quite difficult.

However, while the path I took throughout this project was full of twists and turns I do believe I have come out of it with a greater and wider understanding of
shaders \& rendering, mesh manipulation, and physically based deformation.
The code from the early projects are located in the "projects\_for\_reflection" folder on the GitHub\footnote{\url{https://github.com/Per-Morten/imt4888_advanced_tech}} repository.

%\include{inc/structure} % could be results
%\chapter{Implementation}
\label{chap:implementation}
\todo{Talk about how the deformation is done without mentioning springs first. I.e. you apply force to it.}
\todo{Also mention the importance of damping.}
\todo{Discuss semi-implicit euler integration scheme. Show math and implementation, discuss advantages and disadvantages with it.}

\section{Semi-Implicit Euler Integration}

\section{Spring Mesh Damping}
The spring mesh damping model is usually presented in a quite intuitive manner. 
Each vertex within a mesh is considered a point with its own mass. 
Between the vertices are springs which try to hold the whole mesh together.
When no forces are applied to the vertices within the model, the length of the springs are in their desired \rephrase{"resting" state.}
When a forces are applied to the vertices and they start moving, the springs holding the vertices together will become stretched
and will try to get back to their resting state.

\todo{Add image}

\section{Implicit Springs - Ball Deformation}
\todo{Add video}
\todo{Add how force is applied, showing the picture from the tutorial}
Jasper Flick\cite{catlike_mesh_deformation} shows an implementation \rephrase{without explicit springs}, following a semi-implicit euler integration scheme. 
In this implementation we keep a copy of the initial vertex positions, as well as a collection of the modified vertices.
Additionally, we keep a collection of the velocity of each vertex, which we update each time force is added to a vertex.
The springs come into play when updating the positions of all the vertices based on their velocities.
Each spring within this system is the vector between the initial positions of the vertices in the mesh and their current positions, as seen in figure \ref{fig:catlike_mesh_deformation_springs}.
Upon update of the vertices the force of a spring is applied to the velocity of a vertex in the direction of the spring, moving the vertex back towards its initial position (see listing~\ref{code:catlike_mesh_deformation_update}).

\begin{figure}
%\centering
    \includegraphics[width=\textwidth]{report/figures/catlike_mesh_deformation_springs.png}
    \caption{Modified vertices(yellow) are pulled towards their original position(blue) by the springs (pink)\cite{catlike_mesh_deformation}}
    \label{fig:catlike_mesh_deformation_springs}
\end{figure}

\begin{figure}
\begin{lstlisting}[label={code:catlike_mesh_deformation_update},language=csharp,caption={Catlike coding mesh deformation vertex update}]
private void UpdateVertex(int i)
{
    Vector3 velocity = mVertexVelocities[i];
    Vector3 spring = mDisplacedVertices[i] - mOriginalVertices[i];
    spring *= UniformScale; 
    velocity -= spring * SpringForce * Time.deltaTime;
    velocity *= 1f - Damping * Time.deltaTime;
    mVertexVelocities[i] = velocity;
    mDisplacedVertices[i] += velocity * (Time.deltaTime / UniformScale);
}
\end{lstlisting}
\end{figure}

\subsection{Usage}
As seen from the video, although simple, the implementation provides quite a lot of flexibility, toying with the different variables 
can lead to some interesting visual effects. 
For example, applying negative force to the sphere can be used to create a visual effect similar to that of a star being swallowed by a black hole.
Applying outward force from the inside of the model can create a relatively convincing effect of something moving around inside the object.
Giving the sphere a high SpringForce will lead to it being harder to deform, and more bouncy when trying to return to its original shape.

\subsection{Limitations}
As previously mentioned the implementation is quite simple, but has some limitations. 
As mentioned by Flick\cite{catlike_mesh_deformation} this implementation is not a physics simulation, while the mesh is deformed
the colliders and physical representation of the object stays the same.
Additionally, none of the vertices are connected to each other through springs, and therefore move completely independent from each other. 
This means that if we were to for instance pin two of the vertices to their initial position, and then apply force to the mesh
none of the other vertices would respect that two of them was locked in place, leading to less than satisfying results.

\section{Verlet Integration}
\todo{Discuss Verlet Integration}

\section{Explicit Springs - Cloth Simulation}
Cloth simulation is a place where the springs are more apparent than the previous implementation.

\subsection{Spring Types}
\subsubsection{Structural Springs}

\subsubsection{Shear Springs}

\subsubsection{Bending Springs}



\todo{We should be able to cut the cloth mesh without any problems right? as long as we "cut" along the triangles, so that no triangle reference a part that has been cut off}



\todo{The technology ahead, ask if this could be used with tesselation and be done on the GPU?}

%\include{inc/discussion}
%\include{inc/conclusion}

\ifthenelse{\boolean{HarvardCitations}}{%
	\bibliographystyle{agsm} % used for Harvard style references. Names - Humanities & Interaction Design
}{%
	\bibliographystyle{ntnuthesis/ntnuthesis} %used for Vancover style references. Numbers - Computer Science & Physics
}

\bibliography{MastersExample}

\appendix
\include{inc/rawdata}
%\include{inc/timetable}

\end{document}
