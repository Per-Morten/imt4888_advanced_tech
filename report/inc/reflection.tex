\chapter{Reflection}
Working on this project has been an enlightening experience, but it has not been without issues.
Originally I wanted to look at shaders in Unity to brush up on my shader knowledge, with the final goal being to end up in the area of terrain deformation.
I spent the beginning of the project toying with Unity's shader graph\footnote{\url{https://blogs.unity3d.com/2018/02/27/introduction-to-shader-graph-build-your-shaders-with-a-visual-editor/}}, creating different visual effects\footnote{Video of some of two of the visual effects in shader graph: \url{https://youtu.be/1DBmfpjmMds}} to verify that my assumptions of how shaders actually worked were correct. 

After getting my feet wet in shader graph I moved over to writing the shaders by hand while looking for tutorials on terrain deformation.
It was hard to find good tutorials on this, especially on the GPU. However, I did stumble across a seemingly good tutorial from Jasper Flick\footnote{\url{https://catlikecoding.com/unity/tutorials/advanced-rendering/surface-displacement/}} 
on surface displacement which seemed quite close to what I wanted to do.
This tutorial assumed that you were already quite familiar with rendering concepts and built on his previous tutorials, so I started going through those.
While working through these tutorials I felt that Unity's shaders were too different from those I had written before in OpenGL, the learning curve
was a bit too steep, debugging tools were quite lacking, and that I was not making good progress towards the end goal.

I looked around for alternative solutions. Writing shaders in Unreal was a possibility, but that requires building the entire engine from source
when you add your own shading models\cite{unreal_shaders}, and the compilation times would quickly hurt my iteration process.
Another alternative was to implement my own rendering program in C++, but that would have taken too much time.

Due to this I decided to move back to the CPU, and work on more general mesh deformation there.
The original intention was to port the logic over to vertex shaders after I had implemented them on the CPU.
I soon moved away from that idea. Implementing stuff on the GPU probably would be more of a chore than a learning experience,
as I already have dealt with data layout and CPU-GPU communication in other courses.
The challenge would be the logic of mesh deformation, not how to do it in shaders.

With the shift over to the CPU, I started to work on procedural mesh generation, as all the different tutorials I looked at
on mesh deformation was working on procedurally generated meshes. Working through the procedural mesh generation tutorials
of Jasper Flick was quite intuitive and gave results, and in the end, I finished his mesh deformation tutorial.
I played around with the implementation for a while, without a clear goal of where to next.

Looking through papers on the topic of mesh deformation most of it seemed to focus on deformation for manipulating meshes more related to creating animations,
which was not really what I was after. 
It was first when I learned that the mesh deformation that Jasper Flick\cite{catlike_mesh_deformation} presented was a sort of mass-spring system
that I figured out that deformation through mass-spring systems would be what I wanted to focus on, and its use in cloth simulation was quite enticing.
Unfortunately, this was quite late in the course, so I did not get to spend as much time on the background of the technology, alternatives
and the future of the technology as I originally wanted.
I would have loved to get more into the other approaches like the finite element methods or free form deformations or try the mass-spring system
within other contexts or with other integrators, moving more into the area of physics simulations.
Figuring out the topic so late also meant that I ran into the problem that you do not really know what to search for until you have been in an area for a while,
which meant that looking around for more background papers and tutorials was quite difficult.

However, while the path I took throughout this project was full of twists and turns I do believe I have come out of it with a greater and wider understanding of
shaders \& rendering, mesh manipulation, and physically based deformation.

The code from the early projects are located in the "projects\_for\_reflection" folder on the GitHub\footnote{\url{https://github.com/Per-Morten/imt4888_advanced_tech}} repository.
