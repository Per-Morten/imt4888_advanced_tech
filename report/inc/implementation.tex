\chapter{Implementation}
\label{chap:implementation}
This has the description of how you actually went about implementing the project.  This should be focused on the interesting challenges and how those related to the project.

There is often the need to layout tables that are long.  We suggest using the sidewaystable mode for this.  You will need to manage the total width of the table to make it fit, but you can see how to do that in the example in Table~\ref{tbl:CBM-MC}.

Writing your own \LaTeX{} tables is very slow, and error prone. There are many converstion tools to help create tables.  The one we used for the sideways table, Table~\ref{tbl:CBM-MC} is LatexKit\footnote{Google plugin for LatexKit \url{https://chrome.google.com/webstore/detail/latexkit/piadpbgaacpbaicjilhfebbfgofomiic}}



% you can also set widths using a new command 
%\newcommand{\Colwidth}{0.08\textwidth}
%\begin{tabular}{l|p{\Colwidth}|p{\Colwidth}|p{\Colwidth}|p{\Colwidth}|p{\Colwidth}|p{\Colwidth}|p{\Colwidth}|p{\Colwidth}|}

\begin{sidewaystable}
\small{
\begin{tabular}{l|cccccccccccccccccc}
Candidate & 1 & 2 & 3 & 4 & 5 & 6 & 7 & 8 & 9 & 10 & 11 & 12 & 13 & 14 & 15 & Num & CBM & Diff. \\
\hline
10001 & 0 & -0.5 & 1 & -0.5 & 1 & -0.5 & -0.5 & 2 & 2 & -2 & 2 & 1.5 & -0.5 & 2 & 2 & 8 & 9 & 0.5 \\
10002 & 0 & 2 & 0 & 0 & 0 & 2 & 2 & 2 & 1 & 1 & 1 & 2 & 0 & 2 & 2 & 10 & 17 & 4.5 \\
10003 & 0 & 2 & -0.5 & 0 & 0 & 2 & 1.5 & 1.5 & 2 & -0.5 & 2 & 1 & 0 & 1.5 & 1.5 & 9 & 14 & 3.5 \\
10004 & 2 & -0.5 & 1 & 0 & 0 & 0 & 2 & 2 & 1 & 0 & 0 & 1 & -0.5 & 1 & -0.5 & 7 & 8.5 & 1.5 \\
10005 & 1 & 0 & 0 & 0 & 0 & 1 & 1 & 0 & 1 & 1 & 1 & 1 & 0 & 1 & 1 & 9 & 9 & -1.5 \\
10006 & 2 & -2 & 0 & -0.5 & 1.5 & 1 & 0 & 2 & 2 & 0 & 2 & 2 & -0.5 & 2 & 2 & 9 & 13.5 & 3 \\
10007 & 1 & -2 & 0 & 0 & 1 & 0 & 1.5 & 2 & 2 & -2 & 0 & 1.5 & -2 & 2 & 1 & 8 & 6 & -2.5 \\
10008 & 2 & 2 & -2 & 0 & 2 & 0 & 2 & 2 & 1.5 & 1.5 & 2 & 1.5 & -0.5 & 2 & 2 & 11 & 18 & 3.5 \\
10009 & 2 & -0.5 & 1 & 0 & -0.5 & 2 & -0.5 & 2 & 2 & 0 & 2 & 2 & -0.5 & -0.5 & 2 & 8 & 12.5 & 4 \\
10010 & 2 & -2 & 2 & 0 & -2 & 2 & 2 & 2 & 1.5 & 1.5 & 1.5 & 2 & 0 & 2 & 2 & 11 & 16.5 & 2 \\
10011 & 2 & 2 & 0 & 1 & 2 & 2 & 2 & -0.5 & 2 & 0 & 2 & 2 & -0.5 & 2 & 2 & 11 & 20 & 5.5 \\
10012 & 2 & 1 & 0 & 2 & -2 & 2 & 2 & 2 & 2 & 0 & 1 & 2 & 0 & 2 & 2 & 11 & 18 & 3.5 \\
10013 & 2 & 2 & 1.5 & 0 & 0 & 2 & 2 & 2 & 1 & 0 & 1 & 2 & -0.5 & 2 & 2 & 11 & 19 & 4.5 \\
10014 & 2 & 2 & 1 & -0.5 & -2 & -2 & 1.5 & 2 & 2 & 1.5 & 2 & 2 & 2 & -0.5 & 2 & 11 & 15 & 0.5 \\
10015 & 1.5 & -2 & 1.5 & 1.5 & 2 & 0 & 2 & 2 & 1.5 & 1 & 2 & -0.5 & 1 & 0 & 1.5 & 11 & 15 & 0.5 \\
10016 & 1 & 2 & 0 & 0 & 1.5 & 0 & 0 & 2 & 1 & 0 & 1 & 1.5 & -0.5 & 1 & 2 & 9 & 12.5 & 2 \\
10018 & 2 & 2 & 1 & 0 & -2 & 0 & -2 & 1.5 & 2 & 1 & 0 & 1.5 & 0 & 1 & 2 & 9 & 10 & -0.5 \\
10019 & 2 & 2 & -2 & 0 & 1.5 & 2 & -0.5 & -0.5 & 1.5 & 1 & 2 & 2 & 0 & 1 & 2 & 10 & 14 & 1.5 \\
10020 & 2 & 2 & 0 & 0 & 2 & 2 & 2 & 2 & -0.5 & 1 & 2 & 2 & 2 & 2 & 2 & 12 & 22.5 & 4.5 \\
\end{tabular}
}
\caption[Confidence Based Marking]{Multichoice marking example data with 15 questions, Number correct (Num), the Confidence Based Marking (CBM) grade,  and the differentiation of knowledge quality (Diff)}
\label{tbl:CBM-MC}

\end{sidewaystable}

%\begin{sidewaystable}
%\centering
%  \csvautobooktabular{figures/largeTable.csv}
%\caption[Autogenerated on Sideways page]{using CSV tables to autogenerate the table in a sideways table. This is not an appropriate use of a sideways table as it is small}
%\label{tbl:dataSetsideways}
%\end{sidewaystable}



\todo{add more here. if you are reading this you can see that I am using todo as a way to indicate where the updates should be}


