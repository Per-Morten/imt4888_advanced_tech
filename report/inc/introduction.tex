\chapter{Introduction}
\label{chap:introduction}
Mesh deformation is a broad area, and is done for many different reasons.
It can be used for collision deformation, animation, physics simulation and the likes.
However, mesh deformation can be quite intimidating to approach, due to the mathematical intensity of the field and lack of good resources in layman's terms.

This paper will present different approaches to mesh deformation as visual effects in interactive applications, 
and two implementations for mesh deformation using the mass spring model. 
One being a relatively simple stress ball like deformation, using semi-implicit Euler integration,
the other being a cloth simulation using Verlet integration.
This will be followed by an overview of uses for the implementations, and where the technology might be going in the future.
Following this is a discussion on useful resources and tutorials that can be helpful for people who consider getting into the field.



\think
{
    \begin{itemize}
        \item Done for various reasons: Collision response, animation, physics emulation.
        \item Exist many solutions, with different levels of accuracy
        \item Field can be intimidating due to all the math involved
        \item Many solutions are really accurate, working well in film or non-realtime simulation
        \item Games often need to sacrifice accuracy for speed.
    \end{itemize}
}
