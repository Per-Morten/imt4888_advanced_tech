\chapter{Future}
Predicting the future is always a difficult challenge, especially in fields such as technology and graphics.
However, I would say that it seems like we have settled down on some of the classic conceptual techniques such as spring mass systems, finite element methods and meshless deformations~\cite{muller2005meshless}, at least for interactive applications such as video games. 

I base this on how the mass spring model, although quite old, is still brought up in many of the slides I have seen on cloth simulation from different universities; such as Trinity College in Dublin\cite{mass_spring_cloth_trinity},
and its continuous use in the games industry.
For example the clothing in Alan Wake\footnote{\url{http://www.alanwake.com/}} was done through a mass spring model\cite{alan_wake_mass_spring}, and from my impression this is also the case
in Assassin's Creed Unity\footnote{\url{https://www.ubisoft.com/en-us/game/assassins-creed-unity/}} and Far Cry 4\footnote{\url{https://www.ubisoft.com/en-us/game/far-cry-4/}}\cite{ubisoft_cloth_simulation}.
This might change in the future though as other models are proposed. Weidner et al.~\cite{weidner_eol} proposes a way to deal with the situation where a piece of cloth collides with a straight edges that are not exactly on the vertices leading to sub-realistic results. 
However, it is not entirely clear to me how this implementation actually works and whether or not it is for real-time applications such as games, or would require too much processing power. 
If that is the case, maybe it will be possible in the future.

While other methods do exist I do not think we will move away from mass spring systems very soon, their relative easy of implementation, believably,
and relatively good performance still make them attractive.
It is important to remember that it is not always about having the newest shiniest tool, but rather to use the tool that best fit the situation at hand.
